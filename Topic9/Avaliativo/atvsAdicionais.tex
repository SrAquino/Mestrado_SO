\documentclass{article}
\usepackage{bookmark}

\usepackage[utf8]{inputenc}
\usepackage[brazil]{babel}  % Define o idioma para português do Brasil
\usepackage{csquotes}       % Para lidar com aspas de maneira apropriada com babel
\usepackage{array}
\usepackage{amsmath}
\usepackage{xcolor}

\usepackage{amsmath}
\usepackage{xcolor}
\usepackage{listings}
% Configuração do estilo de código
\lstset{
  language=C,
  %basicstyle=\ttfamily\footnotesize,
  keywordstyle=\color{blue},
  commentstyle=\color{gray},
  stringstyle=\color{red},
  breaklines=true,
  frame=single,
  numbers=left,
  numberstyle=\tiny,
  numbersep=5pt,
  showstringspaces=false,
}


\usepackage{graphicx}       % Pacote para inserção de imagens

\usepackage{hyperref}       % Pacote para links clicáveis
\hypersetup{
    colorlinks=true,
    linkcolor=blue,
    urlcolor=blue,
}

\usepackage{enumitem}       % Pacote para personalização de listas

\usepackage{geometry}       % Pacote para ajustar margens

\geometry{                  % Definindo margens personalizadas (em centímetros)
  left=2.5cm,  % Margem esquerda
  right=2.5cm, % Margem direita
  top=2.5cm,   % Margem superior
  bottom=2.5cm % Margem inferior
}

\usepackage[backend=biber,style=ieee]{biblatex}     % Estilo ieee para bibliografia numerada
\addbibresource{ref.bib}                            % Arquivo .bib


\begin{document}

\begin{titlepage}
    \centering
    % Cabeçalho personalizado
    \includegraphics[width=0.3\textwidth]{../../Topic1/Avaliativo/Imagens/Logo UFLA - Colorida chapada.png}

    \vspace*{2cm} % Espaçamento vertical antes do cabeçalho
    \Large
    Universidade Federal de Lavras\\
    PPGCC\\
    PCC508 – Sistemas Operacionais\\
    
    \vspace{2cm} % Espaço entre o cabeçalho e o título
    \huge % Define o tamanho da fonte do título
    \textbf{Tópico 9 Lista Avaliativa}

    \vfill % Adiciona um espaçamento flexível antes do rodapé (opcional)
    
    % Opcionalmente, você pode incluir seu nome e a data aqui
    \large
    Douglas Aquino T. Mendes\\
    \today % Insere a data atual
\end{titlepage}

\tableofcontents
\newpage

\section{Introdução}
Este documento tem como objetivo apresentar o desenvolvimento das atividades avaliativas para o tópico 9 da disciplina de Sistemas Operacionais, focando na implementação de códigos em linguagem C. Serão apresentadas as questões, a resolução, os códigos desenvolvidos, seguido da apresentação dos resultados da execução do código.

\section{Questões}

\subsection{1}
\textbf{Pergunta:} 1) Explique o que é um driver de dispositivo e como funciona a interface do driver com o sistema operacional.\newline

\textbf{Resposta:} Um driver de dispositivo é um software que permite que o sistema operacional se comunique e controle um dispositivo de hardware específico \parencite[p. 533]{tanenbaum2021}. Ele age como um tradutor entre o sistema operacional, que possui uma visão abstrata do hardware, e o controlador do dispositivo, que possui uma visão de baixo nível e específica do hardware.

A interface do driver com o sistema operacional é definida por um conjunto de funções que o driver deve fornecer. Essas funções permitem que o sistema operacional interaja com o dispositivo de forma abstrata, sem precisar conhecer os detalhes de implementação do driver \parencite[p. 250]{tanenbaum2021}. Por exemplo, um driver de disco pode fornecer funções para ler e gravar blocos de dados, enquanto um driver de teclado pode fornecer funções para ler caracteres digitados. O sistema operacional se comunica com o driver através de chamadas de sistema ou outros mecanismos de comunicação entre o espaço do usuário e o espaço do kernel. O driver, por sua vez, se comunica com o controlador do dispositivo através da E/S mapeada na memória ou portas de E/S \parencite[p. 236]{tanenbaum2021}.

\subsection{2}

\textbf{Pergunta:} 2) Explique a diferença entre a comunicação síncrona e assíncrona. Para um programa de usuário, qual o modelo de comunicação mais conveniente? Explique. \newline

\textbf{Resposta:} A comunicação síncrona ocorre quando o emissor de uma mensagem fica bloqueado, aguardando a resposta do receptor antes de prosseguir com sua execução. É como uma conversa telefônica, onde cada pessoa espera a outra terminar de falar antes de responder. Já na comunicação assíncrona, o emissor envia a mensagem e continua sua execução sem esperar a resposta. É como enviar uma carta pelo correio: você a envia e continua com suas atividades sem esperar uma resposta imediata.

Para programas de usuário, a escolha deve considerar o tipo de interação esperada e o comportamento do sistema. Por exemplo, interfaces altamente interativas, como chats ou jogos multiplayer, podem preferir comunicação síncrona. Sistemas com várias operações simultâneas, como aplicativos de e-commerce ou processamento de grandes volumes de dados, geralmente funcionam melhor com comunicação assíncrona.

\subsection{3}

\textbf{Pergunta:} 3) Explique o software de E/S no nível de usuário, dando uma especial atenção à técnica de spooling. \newline

\textbf{Resposta: }O software de E/S no nível de usuário consiste em bibliotecas e programas que facilitam a interação dos processos com os dispositivos de E/S. Uma parte importante desse software é independente de dispositivo, ou seja, pode ser usada com diversos tipos de dispositivos sem modificações. Uma técnica importante no nível de usuário é o \textbf{spooling}, que permite o compartilhamento de dispositivos de E/S dedicados, como impressoras, em sistemas multiprogramação.
Imagine um sistema onde múltiplos processos precisam imprimir documentos. Sem spooling, se um processo abre o arquivo especial de caractere da impressora e demora para imprimir, outros processos ficam bloqueados, aguardando a liberação do dispositivo. Com o spooling, o sistema operacional cria um diretório de spool onde os processos podem depositar seus trabalhos de impressão. Um daemon de impressão (um processo em segundo plano) é responsável por monitorar o diretório e enviar os trabalhos para a impressora, um de cada vez \parencite[p. 254]{tanenbaum2021}.

\printbibliography % Imprime a lista de referências


\end{document}