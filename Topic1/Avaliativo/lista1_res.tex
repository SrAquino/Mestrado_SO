\documentclass{article}

\usepackage[utf8]{inputenc}
\usepackage[brazil]{babel}  % Define o idioma para português do Brasil
\usepackage{csquotes}       % Para lidar com aspas de maneira apropriada com babel
%\usepackage{ragged2e}       % Pacote para justificação

\usepackage{graphicx}       % Pacote para inserção de imagens

\usepackage{hyperref}       % Pacote para links clicáveis
\usepackage{enumitem}       % Pacote para personalização de listas

\usepackage{geometry}       % Pacote para ajustar margens

\geometry{                  % Definindo margens personalizadas (em centímetros)
  left=2.5cm,  % Margem esquerda
  right=2.5cm, % Margem direita
  top=2.5cm,   % Margem superior
  bottom=2.5cm % Margem inferior
}

\usepackage[backend=biber,style=ieee]{biblatex}     % Estilo ieee para bibliografia numerada
\addbibresource{ref.bib}                            % Arquivo .bib

\begin{document}

\begin{titlepage}
    \centering
    % Cabeçalho personalizado
    \includegraphics[width=0.3\textwidth]{./Imagens/Logo UFLA - Colorida chapada.png}

    \vspace*{2cm} % Espaçamento vertical antes do cabeçalho
    \Large
    Universidade Federal de Lavras\\
    PPGCC\\
    PCC508 – Sistemas Operacionais\\
    
    \vspace{2cm} % Espaço entre o cabeçalho e o título
    \huge % Define o tamanho da fonte do título
    \textbf{Lista Avaliativa 1}
    
    \vfill % Adiciona um espaçamento flexível antes do rodapé (opcional)
    
    % Opcionalmente, você pode incluir seu nome e a data aqui
    \large
    Douglas Aquino T. Mendes\\
    \today % Insere a data atual
\end{titlepage}

\tableofcontents
\newpage

\section{Perguntas}
\subsection{O que é um sistema operacional?}

\textbf{Pergunta:} Explique o que é um sistema operacional, utilizando a visão do sistema como uma máquina estendida e como um gerenciador de recursos.\newline

\textbf{Resposta:} Sistema operacional \textbf{como máquina estendida} é uma abstração para facilitar a comunicação do software com o hardware, tendo em vista que, gerenciar as informações diretamente no hardware é algo muito complexo, então após definir e implementar as abstrações de um SO (Sistema Operacional) o sistema operacional esconde as complexidades do hardware, oferecendo interfaces mais simples e eficientes para os programadores resolverem os problemas. Já como um \textbf{gerenciador de recursos}, o sistema operacional é responsável por alocar e monitorar o uso dos recursos do sistema, como CPU, memória e dispositivos de E/S (entrada e saída). Ele decide quais programas podem acessar determinados recursos, controla o uso desses recursos, contabiliza seu consumo e resolve conflitos de requisição quando há várias solicitações simultâneas.
\cite{ufsm2018}

\subsection{O que o modo supervisor e usuário?}

\textbf{Pergunta:} Explique o que o modo supervisor e usuário, e em que momentos existe a troca de modo do processador.\newline

\textbf{Resposta:} O modo supervisor e o modo usuário são dois estados de execução do processador que permitem ao sistema operacional controlar o acesso aos recursos do sistema e garantir a segurança e estabilidade da execução dos programas. No modo supervisor (também conhecido como modo kernel ou privilegiado), o processador tem acesso irrestrito a todos os recursos do hardware, como a memória, dispositivos de E/S e instruções privilegiadas. Esse modo é usado pelo sistema operacional para realizar operações críticas, como gerenciamento de memória, controle de processos e manipulação direta de hardware. No modo usuário, o processador tem acesso restrito, ou seja, os programas de aplicação podem executar apenas um subconjunto limitado de instruções, sem acesso direto a recursos de hardware sensíveis. Isso previne que falhas ou erros em aplicações comprometam a estabilidade e segurança do sistema \cite{gta_ufrj} \cite{tanenbaum2021}.\newline

A troca de modo do processador ocorre em momentos, como:

\begin{itemize}
    \item \textit{System calls} (chamadas de sistema): Quando um programa em modo usuário precisa de algum serviço do sistema operacional, ele solicita uma chamada de sistema, o que aciona a troca para o modo supervisor para que o SO execute a operação.
    \item Interrupções: Quando um dispositivo de hardware, como um disco ou teclado, precisa da atenção do processador, uma interrupção ocorre, levando o processador a trocar para o modo supervisor para que o sistema operacional trate essa interrupção.
    \item Troca de contexto: Durante a troca entre processos ou threads, o sistema operacional precisa executar operações em modo supervisor para gerenciar a memória e o estado do processador antes de retornar ao modo usuário.
\end{itemize}

\subsection{Quais os passos para a realizar uma chamada de sistema?}
\textbf{Pergunta:} Explique, em detalhes, quais são os passos para a realização de uma chamada de sistema.\newline

\textbf{Resposta:}
\begin{enumerate}
    \item Para obter serviços do sistema operacional, um programa de usuário deve fazer uma chamada de sistema, que, por meio de uma instrução TRAP, chaveia do modo usuário para o modo núcleo e passa o controle para o sistema operacional.
    \item A execução do programa muda de modo usuário para modo kernel. Isso é feito para garantir que o código do sistema operacional tenha controle sobre os recursos do sistema e possa executar operações que não são permitidas em modo usuário.
    \item O número da chamada de sistema, que é geralmente passado em um registrador, é usado pelo kernel para identificar qual serviço deve ser executado.
    \item O kernel verifica a validade dos parâmetros fornecidos.
    \item O kernel executa a operação correspondente à chamada de sistema (como ler um arquivo, criar um processo ou alocar memória).
    \item Após a execução da chamada de sistema, o kernel prepara o resultado da operação, que pode incluir um valor de retorno, como um identificador de arquivo, um ponteiro de memória ou um código de erro.
    \item O sistema operacional muda novamente o modo de execução de volta para modo usuário, permitindo que o aplicativo continue sua execução.
    \item O aplicativo lê o resultado da chamada de sistema, que geralmente está disponível em um registrador ou em um local de memória específico.
    \item O programa retoma sua execução a partir do ponto onde a chamada de sistema foi feita, agora com os resultados da operação. \cite{tanenbaum2021}
\end{enumerate}

\subsection{O que é o sistema de proteção de arquivos?}
\textbf{Pergunta:} Explique o sistema de proteção de arquivos baseado em um código de 9 bits para cada arquivo.\newline

\textbf{Resposta:}
Quando um arquivo é criado, ele recebe o UID e GID do processo criador, além de um conjunto de permissões que determinam o acesso do proprietário, do grupo do proprietário e dos demais usuários. As permissões são organizadas em três categorias: leitura (r), escrita (w) e execução (x). Para arquivos executáveis, a permissão de execução é relevante, e tentar executar um arquivo não executável resultará em erro. Como existem três categorias de usuários e três tipos de permissão por categoria, utiliza-se 9 bits para representar os direitos de acesso \cite{tanenbaum2021}.
Por exemplo, se o dono do arquivo tem permissões de leitura, escrita e execução, o grupo tem permissão de leitura e execução, e outros têm apenas permissão de leitura, o código de 9 bits seria ``rwxr-xr- -''.

\subsection{O que é um processo?}
\textbf{Pergunta:} O que é um processo? Quando um processo não está em execução, onde as informações sobre o mesmo são armazenadas para permitirem que ele continue posteriormente sua execução do mesmo ponto onde foi parada?\newline

\textbf{Resposta:} Um processo é um programa em execução, contendo um espaço de endereçamento que inclui o código executável, dados e pilha. Ele também possui recursos como registradores, lista de arquivos abertos, alarmes e informações necessárias para sua execução. Quando um processo é suspenso, todas essas informações são salvas, permitindo que ele seja retomado do ponto onde parou. Essas informações são armazenadas na tabela de processos, que contém o estado dos registradores e outros dados essenciais para reiniciar o processo posteriormente. O espaço de endereçamento, conhecido como imagem do núcleo, também faz parte desse contexto \cite{tanenbaum2021}.

\subsection{O que é espaço de endereçamento e memória virtual?}
\textbf{Pergunta:} Explique o que é espaço de endereçamento e memória virtual.\newline

\textbf{Resposta: Espaço de Endereçamento} refere-se ao conjunto de endereços de memória que um processo pode acessar. É a faixa de endereços de memória que vai de um valor mínimo (geralmente 0) até um valor máximo, definida pelo sistema operacional e pela arquitetura do hardware. Esse espaço contém tudo o que o processo necessita para executar: o código do programa (instruções), dados (variáveis) e a pilha (para gerenciar chamadas de função e variáveis locais). Cada processo tem seu próprio espaço de endereçamento, isolado dos demais, garantindo segurança e estabilidade no sistema.
E a \textbf{Memória Virtual} é uma técnica de gerenciamento de memória usada pelos sistemas operacionais que proporciona a capacidade de executar programas maiores do que a memória física da máquina, rapidamente movendo pedaços entre a memória RAM e o disco. Na memória virtual, o espaço de endereçamento de um processo é dividido em blocos chamados de páginas, e essas páginas podem ser armazenadas tanto na RAM quanto no disco. Quando uma página que um processo precisa não está na RAM, ocorre uma interrupção chamada page fault, e o sistema operacional carrega a página correspondente do disco para a memória, possibilitando a execução contínua do processo \cite{tanenbaum2021}. 

\subsection{Como estruturar um sistema operacional de micronúcleo?}
\textbf{Pergunta:} Explique uma forma de estruturar um sistema operacional chamada de micronúcleo.\newline

\textbf{Resposta:} Para estruturar um sistema operacional baseado em micronúcleo, a ideia central é dividir o sistema em dois componentes principais, o Micronúcleo que é o núcleo mínimo que executa em modo privilegiado (kernel mode) e gerencia apenas as funcionalidades essenciais do sistema, e os serviços de usuário (que normalmente estariam no núcleo em um sistema monolítico), que operam em modo não privilegiado (user mode).

O micronúcleo deve ser responsável por executar apenas as operações essenciais que exigem acesso direto ao hardware, como, gerenciamento de processos, gerenciamento de memória, comunicação entre processos e gerenciamento de interrupções. As funcionalidades adicionais (como por exemplo, gerenciamento de arquivos, gerenciamento de dispositivos, serviços de rede e  gerenciamento de interface) que em sistemas monolíticos estariam no kernel, são implementadas como processos separados no espaço de usuário. Cada serviço possui responsabilidade limitada e comunicação com outros serviços via IPC. 

Como o micronúcleo executa apenas o mínimo necessário, ele deve fornecer mecanismos eficientes de IPC para que os processos possam trocar informações de forma segura e rápida. E como os serviços são processos de usuário isolados, falhas em um serviço não devem derrubar o sistema como um todo mas o micronúcleo precisa fornecer mecanismos para detectar falhas em serviços de usuário e reiniciar automaticamente um serviço de usuário que falhou sem comprometer o restante do sistema. \cite{tanenbaum2021} \cite{ufpr_sistemas_operacionais} \cite{univasf_microkernel}

\section{Desenvolver um programa}

\subsection{Sorteia um número entre 0 e 200}
\textbf{Enunciado:} Desenvolver um programa onde um processo pai cria cinco filhos. Cada um dos filhos sorteia um número entre 0 e 200 e envia para o pai através de um pipe. O processo pai imprime então qual foi o menor número recebido e também o PID do filho correspondente ao menor número enviado.\newline

\textbf{Resposta:} O código fonte está no arquivo atv8.c e pode ser compilado com o comando "gcc atv8.c -o atv8" e executado com o comando "./atv8"

\subsection{Imprimirá números sequenciais em intervalos de 1 segundo}
\textbf{Enunciado:} Um processo pai irá criar 3 filhos. Cada filho imprimirá na tela números sequenciais em intervalos de 1 segundo, juntamente com o seu PID. O processo pai, após a criação dos filhos, irá permitir que somente 1 filho execute em cada momento por um período de 10s. A cada período, o filho que está executando será parado com SIGSTOP e o próximo será liberado para executar (com SIGCONT). Não esquecer que após a criação dos 3 filhos, todos devem ser parados para dar após início a essa sequência.\newline

\textbf{Resposta:} O código fonte está no arquivo atv8.c e pode ser compilado com o comando "gcc atv9.c -o atv9" e executado com o comando "./atv9"

\printbibliography % Imprime a lista de referências

\end{document}
