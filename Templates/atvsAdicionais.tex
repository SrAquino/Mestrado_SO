\documentclass{article}
\usepackage{bookmark}

\usepackage[utf8]{inputenc}
\usepackage[brazil]{babel}  % Define o idioma para português do Brasil
\usepackage{csquotes}       % Para lidar com aspas de maneira apropriada com babel
\usepackage{array}
\usepackage{amsmath}
\usepackage{xcolor}

\usepackage{amsmath}
\usepackage{xcolor}
\usepackage{listings}
% Configuração do estilo de código
\lstset{
  language=C,
  %basicstyle=\ttfamily\footnotesize,
  keywordstyle=\color{blue},
  commentstyle=\color{gray},
  stringstyle=\color{red},
  breaklines=true,
  frame=single,
  numbers=left,
  numberstyle=\tiny,
  numbersep=5pt,
  showstringspaces=false,
}


\usepackage{graphicx}       % Pacote para inserção de imagens

\usepackage{hyperref}       % Pacote para links clicáveis
\hypersetup{
    colorlinks=true,
    linkcolor=blue,
    urlcolor=blue,
}

\usepackage{enumitem}       % Pacote para personalização de listas

\usepackage{geometry}       % Pacote para ajustar margens

\geometry{                  % Definindo margens personalizadas (em centímetros)
  left=2.5cm,  % Margem esquerda
  right=2.5cm, % Margem direita
  top=2.5cm,   % Margem superior
  bottom=2.5cm % Margem inferior
}

\usepackage[backend=biber,style=ieee]{biblatex}     % Estilo ieee para bibliografia numerada
\addbibresource{ref.bib}                            % Arquivo .bib


\begin{document}

\begin{titlepage}
    \centering
    % Cabeçalho personalizado
    \includegraphics[width=0.3\textwidth]{../../Topic1/Avaliativo/Imagens/Logo UFLA - Colorida chapada.png}

    \vspace*{2cm} % Espaçamento vertical antes do cabeçalho
    \Large
    Universidade Federal de Lavras\\
    PPGCC\\
    PCC508 – Sistemas Operacionais\\
    
    \vspace{2cm} % Espaço entre o cabeçalho e o título
    \huge % Define o tamanho da fonte do título
    \textbf{Tópico 6 Lista Avaliativa}
    
    \vfill % Adiciona um espaçamento flexível antes do rodapé (opcional)
    
    % Opcionalmente, você pode incluir seu nome e a data aqui
    \large
    Douglas Aquino T. Mendes\\
    \today % Insere a data atual
\end{titlepage}

\tableofcontents
\newpage

\section{Introdução}
Este documento tem como objetivo apresentar o desenvolvimento das atividades avaliativas para o tópico 6 da disciplina de Sistemas Operacionais, focando na implementação de códigos em linguagem C. Serão apresentadas as questões, a resolução, os códigos desenvolvidos, seguidos de uma explicação sobre sua lógica de funcionamento.

\section{Questões}

\subsection{1}
\textbf{Pergunta:} 1) \newline

\textbf{Resposta:}  

\subsection{2}

\textbf{Pergunta:} 2)  \newline

\textbf{Resposta:}  

\subsection{3}

\textbf{Pergunta:} 3)  \newline

\textbf{Resposta: }

\subsection{4}

\textbf{Pergunta:} 4)  \newline

\textbf{Resposta:} 

\subsection{5}

\textbf{Pergunta:} 5) \newline

\textbf{Resposta:} 

\section{Desenvolver um programa}

\subsection{}

\textbf{Enunciado:}  \newline

\subsection{Código}
\label{sub-sec-cod}
%\lstinputlisting[language=C]{Codes/atv6T5.c}

\subsection{Testes e Resultados}
Como resultado da execução do código exibido na subceção \ref{sub-sec-cod}, obtivemos a saída ilustrada na figura \ref{fig:exec}. 

\begin{figure}[ht]
    \centering
    %\includegraphics[width=1\textwidth]{./Images/saída.png}
    \caption{Resultado da execução do programa}
    \label{fig:exec}
\end{figure}

\printbibliography % Imprime a lista de referências


\end{document}