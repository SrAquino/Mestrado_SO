\documentclass{article}
\usepackage{bookmark}

\usepackage[utf8]{inputenc}
\usepackage[brazil]{babel}  % Define o idioma para português do Brasil
\usepackage{csquotes}       % Para lidar com aspas de maneira apropriada com babel
\usepackage{array}
\usepackage{amsmath}
\usepackage{xcolor}

\usepackage{amsmath}
\usepackage{xcolor}
\usepackage{listings}
% Configuração do estilo de código
\lstset{
  language=C,
  %basicstyle=\ttfamily\footnotesize,
  keywordstyle=\color{blue},
  commentstyle=\color{gray},
  stringstyle=\color{red},
  breaklines=true,
  frame=single,
  numbers=left,
  numberstyle=\tiny,
  numbersep=5pt,
  showstringspaces=false,
}


\usepackage{graphicx}       % Pacote para inserção de imagens

\usepackage{hyperref}       % Pacote para links clicáveis
\hypersetup{
    colorlinks=true,
    linkcolor=blue,
    urlcolor=blue,
}

\usepackage{enumitem}       % Pacote para personalização de listas

\usepackage{geometry}       % Pacote para ajustar margens

\geometry{                  % Definindo margens personalizadas (em centímetros)
  left=2.5cm,  % Margem esquerda
  right=2.5cm, % Margem direita
  top=2.5cm,   % Margem superior
  bottom=2.5cm % Margem inferior
}

\usepackage[backend=biber,style=ieee]{biblatex}     % Estilo ieee para bibliografia numerada
\addbibresource{ref.bib}                            % Arquivo .bib


\begin{document}

\begin{titlepage}
    \centering
    % Cabeçalho personalizado
    \includegraphics[width=0.3\textwidth]{../../Topic1/Avaliativo/Imagens/Logo UFLA - Colorida chapada.png}

    \vspace*{2cm} % Espaçamento vertical antes do cabeçalho
    \Large
    Universidade Federal de Lavras\\
    PPGCC\\
    PCC508 – Sistemas Operacionais\\
    
    \vspace{2cm} % Espaço entre o cabeçalho e o título
    \huge % Define o tamanho da fonte do título
    \textbf{Tópico 6 Lista Avaliativa}
    
    \vfill % Adiciona um espaçamento flexível antes do rodapé (opcional)
    
    % Opcionalmente, você pode incluir seu nome e a data aqui
    \large
    Douglas Aquino T. Mendes\\
    \today % Insere a data atual
\end{titlepage}

\tableofcontents
\newpage

\section{Introdução}
Este documento tem como objetivo apresentar o desenvolvimento das atividades avaliativas para o tópico 6 da disciplina de Sistemas Operacionais, focando na implementação de códigos em linguagem C. Serão apresentadas as questões, a resolução, os códigos desenvolvidos, seguidos de uma explicação sobre sua lógica de funcionamento.

\section{Questões}

\subsection{Pergunta 1}
\textbf{Pergunta:} 1) Explique como nomes longos de arquivos podem ser manipulados em um diretórios\newline

\textbf{Resposta:}  

\subsection{2}

\textbf{Pergunta:} 2) Como funciona o Journaling em um sistema de arquivos? Explique.\newline

\textbf{Resposta:}  

\subsection{3}

\textbf{Pergunta:} 3) Explique como funciona a alocação de blocos de arquivos baseadas em Inodes.  \newline

\textbf{Resposta: }

\subsection{4}

\textbf{Pergunta:} 4) Explique como funciona uma tabela FAT (File Allocation Table). \newline

\textbf{Resposta:} 

\subsection{5}

\textbf{Pergunta:} 5) Quando um aplicação necessita de um controle mais aprimorado sobre as permissões associadas a um determinado arquivo, as Access Control Lists (ACL) podem ser utilizadas para isso. Descrever uma visão geral de como as ACLs funcionam.\newline

\textbf{Resposta:} 

\section{Desenvolver um programa}

\subsection{Enunciado 6}

\textbf{Enunciado:}  A chamada statvfs(…) é utilizada para obter-se informações sobre um sistema de arquivos montado. Faça um programa que receba como parâmetro o caminho de um sistema de arquivos e apresente as informações obtidas através desta chamada.\newline

\subsubsection{Código}
\label{sub-sec-cod}
%\lstinputlisting[language=C]{Codes/atv6T5.c}

\subsubsection{Testes e Resultados}
Como resultado da execução do código exibido na subceção \ref{sub-sec-cod}, obtivemos a saída ilustrada na figura \ref{fig:exec}. 

\begin{figure}[ht]
    \centering
    %\includegraphics[width=1\textwidth]{./Images/saída.png}
    \caption{Resultado da execução do programa}
    \label{fig:exec}
\end{figure}

\subsection{Enunciado 7}

\textbf{Enunciado:}   Criar dois programas. O primeiro cria um arquivo binário, com 30 ints, realizando uma contagem (1,2,3,4...30). Isso significa que os 30 ints devem ser armazenados em sequência no arquivo. O segundo programa deve utilizar a função readv para fazer a leitura simultânea em múltiplos buffers de 8 números do arquivo gerado pelo primeiro programa. Essa leitura deve ser feita somente com uma instrução readv. Os números a serem lidos devem ser a partir do décimo armazenado (décimo no buffer 1, décimo primeiro no buffer 2, e assim por diante). Imprimir os números na tela. Não esqueça que cada int tem o seu tamanho em bytes fixo.\newline

\subsubsection{Código}
\label{sub-sec-cod}
%\lstinputlisting[language=C]{Codes/atv6T5.c}

\subsubsection{Testes e Resultados}
Como resultado da execução do código exibido na subceção \ref{sub-sec-cod}, obtivemos a saída ilustrada na figura \ref{fig:exec}. 

\begin{figure}[ht]
    \centering
    %\includegraphics[width=1\textwidth]{./Images/saída.png}
    \caption{Resultado da execução do programa}
    \label{fig:exec}
\end{figure}

\printbibliography % Imprime a lista de referências


\end{document}